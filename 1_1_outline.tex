\secrel{Section 1: An Outline of Basic Ideas\\Краткое изложение основных идей}

Three centures ago science was transofrmed by the dramatic new idea
that rules based on mathematical equations could be used to describe
the natural world. My purpose in this book is to initiate another such
transformation, and to intruduce a new kind of science that is based on
the much more general types of rules that can be embodied in simple
computer programs.

It has taken me the better part of twenty years to build the
intellectual structure that is nedded, but I have been amazed by its
results. For that I have found is that with the new kind of science I
have developed it suddenly becomes possible to make progress on a
remarkable range of fundamental issues that have never successfully
been addresed by any of the existing sciences before.

If theoretical science is to be possible at all, then at some level
the systems it studies must follow definite rules. Yet in the past
throughout the exact sciences it has usually been assumed that these
rules must be ones based on traditional mathematics. But the crucial
realization that led me to develop the new kind of science in this book
is that there is in fact no reason to think that systems like those we see
in nature should follow only such traditional mathematical rules.

Erarlier in history it might have been difficult to imagine what
more general types of rules could be like. But today we are surrounded
by computers whose programs in effect implement a huge variety of
rules. The programs we use in practive are mostly based on extremely
compilcated rules specifically designed to perform particular tasks. But
a program can in principle follow essentially any definite set of rules.
And at the core of the new kind of science that I describe in this book
asre discoveries I have made about programs with some of the very
simplest rules that are possible.

One might have thought\ --- as at first I certainly did\ --- that if the
rules for a program were simple then this would mean that its behavior
must also be correspondingly simple. For out everyday experiance in
building things tends to give us the intuitium that creating complecity is
somehow difficult, and requires rules or plans that are themselves
complex. But the pivotal discovery that I made some eighteen years ago is
that in the world of programs such intuition is not even close to correct.

I did what is in a sense one of the most elementary imaginable
computer experimets: I took a sequence of simple programs and then
systematically ran them to see how they behaved. And what I found\ ---\ 
to my great surprice\ --- was that dispite the simplicity of their rules, the
behavior of the programs was often far from simple. Indeed, even some
of the very simplest programs that I looked at had behavior that was as
complex as anything I had ever seen.

It took me more then a decade to come to terms with this result,
and to realize just how fundamental and far-reaching its consequences
are. In retrospect there is no reason the result could not have been found
centureies ago, but increasingly I have come to view it as one of the more
important single discoveries in the whole history of theretical science.
For in addition to opening up vast new domainds of exploration, it implies
a radical rethinking of how processes in nature and elsewhere work.

Perhaps immediately most dramatic is that it yields a resolution
to what has long been considered the single greatest mystery of the
natural world: what secret it is that allows nature seemingly so
effortlessly to produce so much that appears to us so complex.

It could have been, after all, that in the natural world we would
mostly see forms like squares and circles that we consider simple. But
in fact one of the most striking features of the natural world is that
across a vast range of physical, biological and other systems we are
continually confronted with what seems to be immense complexity.
And indeed throughout most of history it has been taken almost for
granted that such complexity\ --- being so vastly greater than in the works
of humans\ --- could only be the work of a supernatural being.

But my discovery that many very simple programs produce great
complexity immediately suggests a rather different explanation. For all
it takes is that systems in nature operate like typical programs and then
it follows that their behavior will often be complex. And the reason that
such complexity is not usually seen in human artifacts is just that in
building these we tend in effect to use programs that are specially
chosen to give only benavior simple enought for us to be able to see that
it will achieve the purposes we want.

One might have thought that with all their successes over the
past few centuries the existing sciences would long ago have managed
to address the ussue of complexity. But in fact thay have not. And indeed
for the most part they have specifically defines their scope in order to
avoid direct contact with it. For while their basic idea of desciribing
behavior in terms of mathematical equations works well in cases like
planetary motion where the behavior is fairly simple, it almost
inevitably fails whenever the behavior is more complex. And more or
less the same is true of descriptions bases on ideas like natural selection
in biology. But by thinking in terms of programs the new kind of
science that I develop in this book is for the first time able to make
meaningful statements about even immensely complex behavior.

In the existing sciences much of the emphasis over the past
century or so has been on breaking systems down to find their
underlying parts, then trying to analize these parts in as much detail as
possible. And particularly in physucs this approach has been sufficiently
successful that the basic components of everyday systems are by now
completely known. But just how these components act together to
produce even some of the most obvios features of the overall behavior
we senn has in the past remained an almost complete mystery. Within
the framework of the new kind of science that I develop in this book,
however, it is finally possible to address such a question.

From the tradition of the existing sciences one might expect that
its answer would depend on all sorts of details, and be quite different for
different types of physical, biological and other systems. But in the
world of simple programs I have discovered that the same basic forms of
behavior occur over and over again almost independent of underlying
details. And what this suggests is that there are quite universal
principles that determine overall behavior and that can be expected to
apply now only to simple programs but also to systems throughout the
natural world and elsewhere.

In the existing sciences whenever a phenomenon is encountered
that seems complex it is taken almost for granted that phenomenon
must be the result of some underlying mechanism that is itself
complex. Mut by discovery that simple programs can produce great
complexity makes it clear that this is not in fact correct. And indeed in
the later parts of this book I will show that even remarkably simple
programs seem to capture the essential machanisms responsible for all
sorts of important phenomena that in the past have always seemed far
too complex to allow any simple explanation.



