\secly{Предисловие}

Just over twenty years ago I made what at first seemed like a small
discovery: a computer experiment of mine showed something I did not expect.
\ru{Чуть более двадцати лет назад я сделал то, что на первый взгляд казалось
маленьким открытием: мой компьютерный эксперимент показал что-то, чего 
я не ожидал.}
But the more I investigated, the more I realized that what I had seen was the
beginning of a crack in the very foundations of existing science, and a first
clue towards a whole new kind of science. \ru{Но чем больше я работал, тем
больше я понимал, что то, что я видел, было началом разлома самих основ
существующей науки, и первым шагом в направлении целого нового направления
науки.}

This book is the culmination of nearly twenty years of work that
I have done to develop that new kind of science.
\ru{Эта книга является кульминацией почти двадцатилетней работы, которые я
провел в развитии этого нового вида науки}.
I had never expected it
would take anything like a long, but I have discovered vastly more
than I ever thought possible, and in fact what I have done now touches
almost every existing area of science, and quite a bit besides.
\ru{Я никогда не ожидал что это займет столько времени, но я внезапно обнаружил
много больше чем я ожидал, и фактически то что я теперь делаю, касается каждой
существующей области науки, и даже немного больше}.

In the early years, I did as I had done before as a scientist, and
published accounts of my ongoing work in the scientific literature.
\ru{В первые годы я делал то же, что делал всегда как ученый\ --- публиковал
отчеты о моей текущей работе в научных изданиях}.
But
although what I wrote seemed to be very well received, I gradually came
to realize that technical papers scattered across the journals of all sorts of
fields could never successfully communicate the kind of major new
intellectual structure that I seemed to be beginning to build.
\ru{Но хотя то что я написал, очень хорошо принималось, я постепенно пришел
к пониманию того, что технические статьи, разбросанные по журналам
посвященным разнообразным областям знания, не могут успешно взаимодействовать
между собой, в отличие от новой интеллектуальной структуры которую кажется
я уже начал строить}.

So I resolved to keep working quietly until I had finished, and
was ready to present everything in a single coherent way. 
\ru{Так что я решил тихо работать, пока не был готов представить все в
едином согласованном виде}.
Fifteen years
later this book is the result.
\ru{Через пятнадцать лет результатом стала эта книга}.
And with it my hope is to share what I have
done with as wide a range of scientists and non-scientist as possible.
\ru{И с помощью нее я надеюсь поделиться тем, что я сделал, с максимально широким
кругом ученых и недоученых}.

In modern times it has been almost unheard of for genuinely new
science to be presented for the first time in a book that can be read by
non-scientists.
\ru{В наше время почти неслыханно, чтобы подлинно новая наука была впервые
представлена в книге, которая может быть прочитана неученым}.
For progress in science has mostly tended to take place
in small steps that cannot reasonably be explained without relying on
specialized technical knowledge of what has gone before.
\ru{Для прогресса в науке в основном предпринимаются небольшие шаги, которые
не могут быть разумно объяснены без опоры на специальные технические знания,
существовавшие ранее}.

But to develop the new kind of science that I describe in this book I
have had no choice but to take several large steps at once, and in doing so
I have mostly ended up having to start from scratch\ --- with new ideas and
new methods that ultimately depend very little on what has gone before.
\ru{Но в процессе создания нового вида науки, который я описал в этой книге,
у меня не было никаких вариантов, кроме как сразу предпринять несколько больших
шагов, и делая это, я в конечном итоге был вынужден начать с нуля\ --- с новыми
идеями и методами, которые крайне мало опираются на предыдущие знания}.

In some ways it might have been easier for me to present what I
have done in some kind of new technical formalism.
\ru{В какой-то мере для меня возможно было бы легче представить то, что я
сделал, в форме некоторого технического формализма}.
But instead I have
chosen to spend the effort to take things to the point where they are clear
enough to be explained quite fully just in ordinary language and pictures.
\ru{Но вместо этого я решил потратить дополнительные усилия, чтобы
рассмотреть мою работу с точки зрения, достаточно ясной, чтобы объяснения
были вполне полными даже в форме текста на обычном языке и иллюстраций}.

