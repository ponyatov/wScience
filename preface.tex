\secly{Предисловие}

Just over twenty years ago I made what at first seemed like a small discovery: a
computer experiment of mine showed something I did not expect.
\ru{Чуть более двадцати лет назад я сделал то, что на первый взгляд казалось
маленьким открытием: мой компьютерный эксперимент показал что-то, чего я не
ожидал.}
But the more I investigated, the more I realized that what I had seen was the
beginning of a crack in the very foundations of existing science, and a first
clue towards a whole new kind of science.
\ru{Но чем больше я работал, тем больше я понимал, что то, что я видел, было
началом разлома самих основ существующей науки, и первым шагом в направлении
целого нового направления науки.}

This book is the culmination of nearly twenty years of work that I have done to
develop that new kind of science.
\ru{Эта книга является кульминацией почти двадцатилетней работы, которые я
провел в развитии этого нового вида науки}.
I had never expected it would take anything like a long, but I have discovered
vastly more than I ever thought possible, and in fact what I have done now
touches almost every existing area of science, and quite a bit besides.
\ru{Я никогда не ожидал что это займет столько времени, но я внезапно обнаружил
много больше чем я ожидал, и фактически то что я теперь делаю, касается каждой
существующей области науки, и даже немного больше}.

In the early years, I did as I had done before as a scientist, and published
accounts of my ongoing work in the scientific literature.
\ru{В первые годы я делал то же, что делал всегда как ученый\ --- публиковал
отчеты о моей текущей работе в научных изданиях}.
But although what I wrote seemed to be very well received, I gradually came to
realize that technical papers scattered across the journals of all sorts of
fields could never successfully communicate the kind of major new intellectual
structure that I seemed to be beginning to build.
\ru{Но хотя то что я написал, очень хорошо принималось, я постепенно пришел к
пониманию того, что технические статьи, разбросанные по журналам посвященным
разнообразным областям знания, не могут успешно взаимодействовать между собой, в
отличие от новой интеллектуальной структуры которую кажется я уже начал
строить}.

So I resolved to keep working quietly until I had finished, and was ready to
present everything in a single coherent way.
\ru{Так что я решил тихо работать, пока не был готов представить все в едином
согласованном виде}.
Fifteen years later this book is the result.
\ru{Через пятнадцать лет результатом стала эта книга}.
And with it my hope is to share what I have done with as wide a range of
scientists and non-scientist as possible.
\ru{И с помощью нее я надеюсь поделиться тем, что я сделал, с максимально
широким кругом ученых и недоученых}.

In modern times it has been almost unheard of for genuinely new science to be
presented for the first time in a book that can be read by non-scientists.
\ru{В наше время почти неслыханно, чтобы подлинно новая наука была впервые
представлена в книге, которая может быть прочитана неученым}.
For progress in science has mostly tended to take place in small steps that
cannot reasonably be explained without relying on specialized technical
knowledge of what has gone before.
\ru{Для прогресса в науке в основном предпринимаются небольшие шаги, которые не
могут быть разумно объяснены без опоры на специальные технические знания,
существовавшие ранее}.

But to develop the new kind of science that I describe in this book I have had
no choice but to take several large steps at once, and in doing so I have mostly
ended up having to start from scratch\ --- with new ideas and new methods that
ultimately depend very little on what has gone before.
\ru{Но в процессе создания нового вида науки, который я описал в этой книге, у
меня не было никаких вариантов, кроме как сразу предпринять несколько больших
шагов, и делая это, я в конечном итоге был вынужден начать с нуля\ --- с новыми
идеями и методами, которые крайне мало опираются на предыдущие знания}.

In some ways it might have been easier for me to present what I have done in
some kind of new technical formalism.
\ru{В какой-то мере для меня возможно было бы легче представить то, что я
сделал, в форме некоторого технического формализма}.
But instead I have chosen to spend the effort to take things to the point where
they are clear enough to be explained quite fully just in ordinary language and
pictures.
\ru{Но вместо этого я решил потратить дополнительные усилия, чтобы рассмотреть
мою работу с точки зрения, достаточно ясной, чтобы объяснения были вполне
полными даже в форме текста на обычном языке и иллюстраций}.

Unfortunately, however, this will no doubt mean that there are some\ ---
particularly from the existing sciences\ --- who will at first assume that their
existing technical knowledge must somehow already cover whatever is in this
book.
\ru{Однако, к сожалению это несомненно будет означать, что будет множество
людей\ --- в частности, представителей существующих наук\ --- кто сразу будет
предполагать что их существующие знания уже охватывают все, что описано в этой
книге}.
And a few, I fear, will stop at that point, and choose to learn no more.
\ru{И я боюсь, что некоторые из них остановятся в этой точке, и выберут путь
отказа от новых знаний}.
But many, I hope, will at least look at the book long enough to begin to be
surprised by what is actually says.
\ru{Но многие, я надеюсь, по крайней мере посмотрят на эту книгу достаточно
долго, чтобы начать удивляться тому, что она реально говорит}.

At first probably they will think that arts of it cannot possibly be correct\
--- for they seem so at odds with existing science.
\ru{Сначала, вероятно, они будут думать что эти трюки не могут быть правильными\
--- потому что кажется что они сильно расходятся с существующей наукой.}.
And indeed if I myself were just to pick up this book today without having spent
the post twenty years thinking about its contents, I have little doubt that I
too would not believe many of the things it says.
\ru{И в самом деле, если бы я сам взял сегодня впервые эту книгу, не потратив
последние двадцать лет в раздумьях над ее содержимым, несомненно я тоже не
поверил бы многим вещам, которая она говорит}.

But the computer experiments on which the science in the book is ultimately
based are easy to check on any modern computer.
\ru{Но идеи науки, на которой базируется эта книга, легко проверить в
эксперименте на любом современном компьютере}.
And almost all the arguments in the book\ --- while often not conceptually
simple\ --- require no specialized scientific or other knowledge to follow.
\ru{И почти все аргументы в книге\ --- даже часто непростые концептульно\ --- не
требуют какого-то специального научного или другого знания, чтобы следовать
им.}.

Yet it has certainly taken me years to come to terms with the conclusions I have
reached.
\ru{Мне конечно постребовали годы, чтобы прийти в согласие с выводами, которых я
достиг}.
And while I hope that all the effort I have put into presentation in this book
will make it easier for others, I do not expect it to be a quick process.
\ru{И хотя я надеюсь, что все усилия, которые я предпринял в этой книге, сделают
ее более легкой для других, я не ожидаю что это будет быстрый процесс}.
For to absorb in any real way what the book has to say requires a fairly major
shift in intuition and thinking.
\ru{Для принятия того пути, который предлагает книга, нужно скачать потребуется
довольно серьезный сдвиг в интуиции и мышлении}.

But the most important first step, I believe, is just to recognize what is
involved.
\ru{Но самый важный первый шаг, я считаю, просто распознать что именно нас
запутывает}.
For thought there are connections of all sorts, this book is first and foremost
about a fundamentally new intellectual structure, that needs to be understood in
its own terms, and cannot reasonably be fit into any existing framework.
\ru{Мышление по сути есть связи всех видов, и эта книга в первую очередь
посвящена принципиально новой интеллектуальной структуре, которая должна быть
понята в ее собственных терминах, и не может приемлемо вписаться в какую-либо
существующую систему представлений}.

It has been a great challenge for me to capture the things I have discovered
over the past twenty years in a book of manageable size.
\ru{Для меня стоило больших усилий запихать то, что я открыл за последние
двадцать лет, в рамки книги приемлемого размера}.
And to do so I have often ended up compressing into a page or even a paragraph
the essence of what a chapter or even a book could have been written about.
\ru{И чтобы сделать это, мне часто приходилось умещать на странице или даже в
одном параграфе то, чему самому по себе может быть посвящена глава или даже
отдельная книга.}

In the quarter million or so words of the main text my emphasis is on
communicating the core of my ideas and discoveries\ --- as well as indicating a
little of how I came to them.
\ru{В примерно четверти миллиона слов основного текста я сконцентрировался на
передаче ядра моих идей и открытий\ --- а также немного отметил, как я пришел к
ним.}
The last three hundred or so pages of the book\ --- themselves another quarter
million or so words\ --- supplement the main text with many historical and
technical notes, and also summarize more discoveries.
\ru{Последние примерно триста страниц книги\ --- еще около четверти миллиона
слов\ --- дополняют основной текст большим количеством исторических и
технических замечаний, и также включают больше открытий.}
The notes that begin on page \pageref{p849} address some specific issues about
reading this book.
\ru{Записи, начинающиеся со страницы \pageref{p849}, посвящены некоторым
специальным замечаниям о чтении этой книги.}

Throughout the book my primary concern is with basic science and fundamental
issues.
\ru{На протяжении всей книги моей главной темой является базовая наука и ее
фундаментальные вопросы.}
But building on the foundations in the book there are a vast array of
applications\ --- both conceptual and practical\ --- that can now be developed.
\ru{Но на фундаменте этой книги возможно построение огромного множества
приложений\ --- как концептуальных, так и практических\ --- которые теперь могут
быть разработаны.}

No doubt some will come quickly. But most will probably take decades to emerge.
\ru{Без сомнений некоторые придут к принятию этих идей быстро. Но большинству
возможно потребуются десятилетия, чтобы присоединиться.}
Yet in time I expect that the ideas of this book will come to pervade not only
science and technology but also many areas of general thinking.
\ru{Тем не менее со временем я ожидаю что идеи этой книги не только будут
пронизывать будущую науку и технологию, но также и многие области мышления
вобще.}
And with this its methods will eventually become a standard part of education\
--- much as mathematics is today.
\ru{И в конечном итоге эти методы станут частью стандартного образования\ ---
так же как математика сегодня.}
And in the end most of what now seems surprising and remarkable in the book will
come to seem familiar and commonplace.
\ru{B результате большая часть того, что сейчас кажется удивительным и
замечательным в книге, станет казаться привычным и обыденным.}

But for me what has always been most important is the actual process of
discovery.
\ru{Но для меня всегда был важным процесс открытия сам по себе.}
For I know of nothing as profoundly exciting as to glimpse for the first time
some new and basic truth.
\ru{Ибо я не знаю ничего настолько захватывающего, чем заглянуть в первый раз в
некоторые новые и базовые истины.}
And now that I have finished building the intellectual structure that I describe
in this book it is my hope that those who read these words can share in the
excitement I have had in making the discoveries that were involved.
\ru{И теперь, когда я закончил построение интеллектуальной структуры, которую я
описал в этой книге, я надеюсь что те, кто читают эти слова, cмогут испытать то
же волнение, которое я испытавал совершая открытия, и присоединятся.}

\bigskip
\hfill Stephen Wolfram

\ru{\hfill Степан Вольфрам}

\hfill January 15, 2002

\ru{\hfill 15 января 2002 года}
