\secly{Предисловие}

Just over twenty years ago I made what at first seemed like a small
discovery: a computer experiment of mine showed something I did not expect.
\ru{Чуть более двадцати лет назад я сделал то, что на первый взгляд казалось
маленьким открытием: мой компьютерный эксперимент показал что-то, чего 
я не ожидал.}
But the more I investigated, the more I realized that what I had seen was the
beginning of a crack in the very foundations of existing science, and a first
clue towards a whole new kind of science. \ru{Но чем больше я работал, тем
больше я понимал, что то, что я видел, было началом разлома самих основ
существующей науки, и первым шагом в направлении целого нового направления
науки.}

This book is the culmination of nearly twenty years of work that
I have done to develop that new kind of science.
\ru{Эта книга является кульминацией почти двадцатилетней работы, которые я
провел в развитии этого нового вида науки}.
I had never expected it
would take anything like a long, but I have discovered vastly more
than I ever thought possible, and in fact what I have done now touches
almost every existing area of science, and quite a bit besides.
\ru{Я никогда не ожидал что это займет столько времени, но я внезапно обнаружил
много больше чем я ожидал, и фактически то что я теперь делаю, касается каждой
существующей области науки, и даже немного больше}.

In the early years, I did as I had done before as a scientist, and
published accounts of my ongoing work in the scientific literature.
\ru{В первые годы я делал то же, что делал всегда как ученый\ --- публиковал
отчеты о моей текущей работе в научных изданиях}.
But
although what I wrote seemed to be very well received, I gradually came
to realize that technical papers scattered across the journals of all sorts of
fields could never successfully communicate the kind of major new
intellectual structure that I seemed to be beginning to build.
\ru{Но хотя то что я написал, очень хорошо принималось, я постепенно пришел
к пониманию того, что технические статьи, разбросанные по журналам
посвященным разнообразным областям знания, не могут успешно взаимодействовать
между собой, в отличие от новой интеллектуальной структуры которую кажется
я уже начал строить}.


